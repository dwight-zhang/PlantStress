\documentclass[11pt]{amsart}
%     If you need symbols beyond the basic set, uncomment this command.
\usepackage{amssymb}
\usepackage{amsmath}
\usepackage{mathrsfs}
\usepackage{bm}
\numberwithin{figure}{section}

%     If your article includes graphics, uncomment this command.
\usepackage{graphicx}

%     If the article includes commutative diagrams, ...
\usepackage[cmtip,all]{xy}

%     If the article includes some tikzpictures.
\usepackage{tikz}

%     Set each margin to 1in
\usepackage[margin=1in]{geometry}

%     Algorigthm
\usepackage{algpseudocode}
\usepackage{algorithmicx,algorithm}

%     Define theorem, definition, example, exercise, equation and remak environments
\theoremstyle{plain}
\newtheorem{theorem}{Theorem}[section]
\theoremstyle{definition}
\newtheorem{lemma}[theorem]{Lemma}
\newtheorem{defi}[theorem]{Definition}
\newtheorem{xca}[theorem]{Exercise}
\newtheorem{sol}[theorem]{Solution}
\newtheorem*{remark}{Remark}
\numberwithin{equation}{section}

%    Blank box placeholder for figures (to avoid requiring any
%    particular graphics capabilities for printing this document).
\newcommand{\blankbox}[2]{%
  \parbox{\columnwidth}{\centering
%    Set fboxsep to 0 so that the actual size of the box will match the
%    given measurements more closely.
    \setlength{\fboxsep}{0pt}%
    \fbox{\raisebox{0pt}[#2]{\hspace{#1}}}%
  }%
}

%    To combine rows in talbe
\usepackage{multirow}

%    Generating simple two- and three-set Venn diagrams 
\usepackage{venndiagram}

%    Set itemize 
\usepackage{enumitem}
\setenumerate[1]{leftmargin=2em,itemsep=0pt,partopsep=0pt,parsep=\parskip,topsep=5pt}
\setitemize[1]{leftmargin=2em,itemsep=0pt,partopsep=0pt,parsep=\parskip,topsep=5pt}
\setdescription[1]{leftmargin=2em,itemsep=0pt,partopsep=0pt,parsep=\parskip,topsep=5pt}

%    Private Package
\usepackage{examplepackage}

%    body
\begin{document}
 
%    Set noindent for entire file
\pagestyle{plain}

%    Set parskip for entire file
\setlength\parskip{1em}

%    Information for title
\title{}

%    Information for author
\author{}

\date{}

% \maketitle

\section{Stress feedback model}
Assume cell wall is a smooth surface $S \subset \mathbf{R}^{3} $, $\bm{n}$ is the exterior normal vector to $S$.
Define the deformation is $\bm{u}$, its tangential gradient is denoted by $\nabla_{S} \bm{u} = \bm{n} \times (\nabla \bm{u} \times \bm{n}) $.
Green-Lagrangian strain tensor is defined by 
$\mathbb{E} = \frac{1}{ 2 } \left(\nabla_{S}\bm{u}^{\top} \nabla_{S}  \bm{u} - \mathbb{I}\right)$.
The motified St. Venant-Kirchhoff model is given by
\begin{align}\label{VK_model}
  \mathcal{E} = \int_S \mathbb{E} : \left(\mathbb{C} \right)\mathbb{E} +  \int_{S} \left( (\nabla \bm{u} - P \mathbb{I})\bm{n}\right)^{2} , 
\end{align}
where $:$ is the double dot product defined by $\mathbb{A}:\mathbb{B} = \text{tr}(\mathbb{A}^{\top} \mathbb{B})$, $\mathbb{C} = \mathbb{C}_{g} + \mathbb{C}_{f} $, 
$\mathbb{C}_{g} , \mathbb{C}_{f} $ respectively are the stiffness matrixes for the gel and fiber,
\[
\mathbb{C}_{g} = Y_{g} \left(
\begin{array}{ccc}
  1 & \nu & 0 \\
  \nu & 1 & 0 \\
  0 & 0 & \frac{ 1-\nu }{ 2 } 
\end{array}
\right), \quad
\mathbb{C}_{f}  = \frac{ \pi Y_{f} \rho_0 }{ 16 } \left(
\begin{array}{ccc}
  3 + \frac{ \rho_2 + 4 \rho_1 }{\rho_0 } & 1- \frac{ \rho_2 }{  \rho_0 } & \frac{ 2\widetilde{\rho_1} + \widetilde{\rho_2}}{\rho_0  }  \\
  1- \frac{ \rho_2 }{  \rho_0 } &  3 + \frac{ \rho_2 - 4 \rho_1 }{\rho_0 } & \frac{ 2\widetilde{\rho_1} - \widetilde{\rho_2}}{\rho_0  }  \\
  \frac{ 2\widetilde{\rho_1} + \widetilde{\rho_2}}{\rho_0  }  & \frac{ 2\widetilde{\rho_1} - \widetilde{\rho_2}}{\rho_0  } &   1- \frac{ \rho_2 }{  \rho_0 } 
\end{array}
\right),
\] 
$P$ is the pressure per unit volume applied to the surface, $\nu,Y_{g} , Y_{f} $ are constants,
$\rho(\theta)$ is $\pi$-periodic angular microfibril distribution function, $\{\widehat{\rho_{n}}(t)\}_{n=0}^{2}  $ are the Fourier transform coefficients given by
\[
\widehat{\rho_{n}}  = \frac{1}{ \pi } \int_{0} ^{\pi} \rho(t, \theta) e^{-2in\theta} d\theta, \quad 
\left(\rho_{n} , \widetilde{\rho_{n} }\right) = 2 \left( \text{Re} \left(\widehat{\rho_{h} }\right), - \text{Im} \left( \widehat{\rho_{n} }\right)\right).
\] 
$\phi(\theta) $ is $\pi$-periodic angular microtubule distribution function.
We have the following evolution and equilibrium equation for angular microfibril and microtubul distribution.
\begin{align}
&  \frac{d \rho (\theta)}{d t}   = k_\rho \frac{\phi(\theta)}{\int_0^\pi \phi(\theta') d \theta'}  - k_\rho^{'}\rho(t, \theta),\\
&  \phi(\theta)   = \frac{c_0 k_\phi e^{\gamma f(\mathbb{C} \mathbb{E}, \theta)}}{1 + k_\phi \int_0^\pi e^{ \gamma f(\mathbb{C}\mathbb{E},\theta')} d \theta'}, 
\quad  f(\mathbb{C}  \mathbb{E}, \theta)  = \bm{e}_\theta^{\top}   \left (\mathbb{C}  \mathbb{E} \right)  \bm{e}_\theta.
\end{align}
where $\mathbb{C} = \mathbb{C}_{g} + \mathbb{C}_{f} $, $c_0, \gamma, k_\phi, k_\rho, k_\rho^{'}$ are suitable positive constants.
By Fourier transform, we have 
\begin{align}
& \frac{ d \widehat{\rho_{n}} }{ d t }  = \frac{ k_{\rho}  }{ \pi } \frac{ \widehat{\phi_{n} } }{ {\phi_{0} } } - k_{\rho} ^{'} \widehat{\rho_{n} },  \quad 
\widehat{\phi_{n}} = \frac{1}{ \pi } \int_{0} ^{\pi} \phi(\theta) e^{-2in\theta} d\theta, \\
& \widehat{\phi_{n}}  = \frac{ c_0 }{ \pi } \frac{ I_{n} \left(2 \gamma \left|\widehat{f_1}\right|\right) }{ { e^{-\gamma f_0} }/( \pi k_{\phi}  ) + I_0 \left( 2 \gamma \left| \widehat{f_1} \right|\right)  } e^{-2in\theta^*}, \quad 
\widehat{f_1} = \frac{ 1 }{ \pi } \int_{0} ^{\pi} f(\mathbb{C}\mathbb{E},\theta) e^{-2i\theta} d \theta,
\end{align}
where $\theta^* $ is the direction of the main stress, $I_{n}$ is the following modified Bessel function of the first kind.
\[
I_{n} (x)  = \frac{ 1 }{ \pi } \int_{0} ^\pi e^{x \cos(\theta)} \cos ( n \theta )d \theta \quad \forall x \in \mathbf{R},
\] 

\newpage 

\section{Weak formulation}

For a bounded surface domain $S \subset \mathbf{R}^{3} $, let the inner product on $L^{2} (S)$ be denoted by 
\[
  (u,v)_{S} = \int_{S} uv,\quad \forall u, v \in L^{2} (S).
\] 
For $\Gamma \subset \partial S$, the sub-space of $H^{1} (S)$ with homogeneous boundary conditions on $\Gamma$ is denoted by
\[
H^{1} _{\Gamma} (S) = \left\{ v \in H^{1} (S): v = 0 \quad \text{on } \quad \Gamma\right\}.
\] 
Moreover, vector-valued and matrix-valed quantities will be denoted by boldface and hollow notations, respectively, such as $\bm{L}^{2} (S) = \left(L^{2} (S)\right)^{3}, \mathbb{L}^{2} (S) = \left(L^{2} (S)\right) ^{3 \times 3} $.

The weak formulation of problem 


\end{document}
%------------------------------------------------------------------------------
% End of tex
%------------------------------------------------------------------------------
