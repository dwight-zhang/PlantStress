%!TEX program = xelatex
\documentclass[9pt,UTF8,aspectratio=43]{beamer}
%% beamer setting 
\usetheme{default}
% Change color
\usecolortheme{dove}
\useinnertheme{circles}
\definecolor{background}{RGB}{255,255,255}
\definecolor{foreground}{RGB}{0,0,0}
\definecolor{title}{RGB}{41,36,33} 
\definecolor{subtitle}{RGB}{24,24,24}
\definecolor{hilit}{RGB}{178,34,34} 
\definecolor{lolit}{RGB}{205,92,92}
\definecolor{uibred}{HTML}{db3f3d}
\definecolor{uibblue}{HTML}{4ea0b7}
\definecolor{uibgreen}{HTML}{789a5b}
\definecolor{uibgray}{HTML}{d0cac2}
\definecolor{uiblink}{HTML}{00769E}
\setbeamercolor{titlelike}{fg=title,bg=uibgray} 
\setbeamercolor{subtitle}{fg=subtitle}
\setbeamercolor{institute}{fg=foreground}
\setbeamercolor{normal text}{fg=foreground,bg=background} 
\setbeamercolor{item}{fg=hilit} % color of bullets
\setbeamercolor{subitem}{fg=lolit} 
\setbeamercolor{itemize/enumerate subbody}{fg=foreground} 
\setbeamertemplate{itemize subitem}{$\Big\circ$} 
\setbeamercolor{block body} {bg = uibgray}
\setbeamercolor{block title}{fg = white,bg = uibred}
\setbeamercolor{block body example} {bg = uibgray}
\setbeamercolor{block title example}{fg = white,bg = uibblue}
\newcommand{\hilit}{\color{hilit}}
\newcommand{\lolit}{\color{lolit}}
% Change font
\usefonttheme{serif}
\usepackage{fontspec}
\setmainfont{Times New Roman}
\setbeamerfont{normal text}{series=\mdseries}
\setbeamerfont{alerted text}{series=\bfseries}
\setbeamerfont{structure}{series=\mdseries}
\setbeamerfont{title}{size=\Large,series=\bfseries}
\setbeamerfont{subtitle}{size=\Large,series=\bfseries}
\setbeamerfont{institute}{size=\normalsize,series=\mdseries}
\setbeamerfont{frametitle}{size=\Large,series=\bfseries}
\setbeamerfont{note page}{size=\footnotesize}
\setbeamerfont{itemize/enumerate subbody}{size=\small} 
\setbeamerfont{itemize/enumerate subitem}{size=\small}
% Slide numbers 
\setbeamertemplate{caption}[numbered] 
\setbeamertemplate{navigation symbols}{}
\setbeamertemplate{footline}{%
\raisebox{5pt}{\makebox[\paperwidth]{\hfill\makebox[20pt]{\hilit \scriptsize\insertframenumber\,/\,\inserttotalframenumber}}}}
% Arrangement
\setbeamercovered{transparent=15}
\setbeamersize{text margin left=0.8cm, text margin right=0.8cm} 

%% include packages 
\usepackage{amsmath,amscd,mathtools,ragged2e,hyperref,verbatim,algorithm,algorithmic,multirow,subfigure,caption,graphicx,bm}
% \captionsetup[figure]{font=footnotesize}
\renewcommand{\thefigure}{}
\renewcommand{\raggedright}{\leftskip=0pt \rightskip=0pt plus 0cm} 
\graphicspath{{figures/}} 
\renewcommand{\thefootnote}{}

%% Title slide setting
\title{A Computational Framework for 3D Mechanical Modeling of Plant Morphogenesis with Cellular Resolution} 
\subtitle{}
\author{
Fr\'ed\'eric Boudon, J\'er\^ome Chopard, Olivier Ali, Benjamin Gilles, Olivier Hamant,
Arezki Boudaoud, Jan Traas, Christophe Godin
}
\institute{}
\date{
  PLOS Computational Biology, (2015) \\
  https://doi.org/10.1007/s00285-018-1286-y
}
 
%% body
\begin{document}
\include{macros}
\raggedright
\setlength{\leftmargini}{8pt}


\begin{frame}{Chen's Example: primordium mesh generator \& simple movement}
\begin{figure}[!htbp]
\centering
\subfigure[stl]{
\includegraphics[width=4.5cm]{./figures/stl}
}
\subfigure[mesh]{
\includegraphics[width=4.5cm]{./figures/initial}
}
\end{figure}
\begin{figure}[!htbp]
\centering
\subfigure[deformation]{
\includegraphics[width=4.5cm]{./figures/move}
}
\subfigure[top]{
\includegraphics[width=4.5cm]{./figures/view}
}
\end{figure}
\footnote{Netgen \& NGSolve: https://ngsolve.org}
\end{frame}

\begin{frame}{Our Stress Feedback Loop Model}
\begin{itemize}
\item Cell wall: smooth surface $S \subset \mathbf{R}^{3} $, $\bm{n}$ is the exterior normal vector to $S$.
\item Deformation: $\bm{u}(\bm{x}) \in \bm{L}^{2} (S)$, tangential gradient: $\nabla_{S} \bm{u} = \bm{n} \times (\nabla \bm{u} \times \bm{n}) \in \mathbb{L}^{2} (S)$.
\item Internal energy for Green-Lagrangian strain tensor $\mathbb{E} = \frac{1}{ 2 } \left(\nabla_{S}\bm{u}^{\top} \nabla_{S}  \bm{u} - \mathbb{I}\right)$: 
\begin{align}
& \mathcal{E}\left[\widehat{\rho_{n} },\nabla \bm{u}\right] = \int_S \mathbb{E} : \left(\mathbb{C}_{g} + \mathbb{C}_{f}  \right)\mathbb{E} +  \int_{S} \frac{ P }{ 3 } \bm{n} \cdot \bm{u}, \\ 
& \mathbb{C}_{g} = Y_{g} \left(
\begin{array}{ccc}
  1 & \nu & 0 \\
  \nu & 1 & 0 \\
  0 & 0 & \frac{ 1-\nu }{ 2 } 
\end{array}
\right), \\
& \mathbb{C}_{f} = \frac{ \pi Y_{f} \rho_0 }{ 16 } \left(
\begin{array}{ccc}
  3 + \frac{ \rho_2 + 4 \rho_1 }{\rho_0 } & 1- \frac{ \rho_2 }{  \rho_0 } & \frac{ 2\widetilde{\rho_1} + \widetilde{\rho_2}}{\rho_0  }  \\
  1- \frac{ \rho_2 }{  \rho_0 } &  3 + \frac{ \rho_2 - 4 \rho_1 }{\rho_0 } & \frac{ 2\widetilde{\rho_1} - \widetilde{\rho_2}}{\rho_0  }  \\
  \frac{ 2\widetilde{\rho_1} + \widetilde{\rho_2}}{\rho_0  }  & \frac{ 2\widetilde{\rho_1} - \widetilde{\rho_2}}{\rho_0  } &   1- \frac{ \rho_2 }{  \rho_0 } 
\end{array}
\right), \\
& \widehat{\rho_{n}} = \frac{1}{ \pi } \int_{0} ^{\pi} \rho(\theta) e^{-2in\theta} d\theta, \quad 
\left(\rho_{n} , \widetilde{\rho_{n} }\right) = 2 \left( \text{Re} \left(\widehat{\rho_{h} }\right), - \text{Im} \left( \widehat{\rho_{n} }\right)\right),
\end{align}
where $:$ is the double dot product defined by $\mathbb{A}:\mathbb{B} = \text{tr}(\mathbb{A}^{\top} \mathbb{B})$, $\mathbb{C}_{g} , \mathbb{C}_{f} $ respectively are the stiffness matrixes for the gel and fiber, $P$ is the pressure per unit volume applied to the surface, $\nu,Y_{g} , Y_{f} $ are constants, $\rho(\theta)$ is angular density of fiber.
\end{itemize}
\end{frame}

\begin{frame}{Our Stres Feedback Loop Model}
\begin{itemize}
\item $\pi$-periodic angular microfibril distribution: $\rho(t,\theta) \in H^{1} _{p} (S)$.
\item $\pi$-periodic angular microtubule distribution: $\phi(\theta) \in H^{1} _{p} (S)$.
\item Evolution and equilibrium equation:
\begin{align}
& \frac{d \rho (t, \theta)}{d t}  = k_\rho \frac{\phi(\theta)}{\int_0^\pi \phi(\theta') d \theta'}  - k_\rho^{'}\rho(t, \theta),\\
& \phi(\theta)  = \frac{c_0 k_\phi \exp(\gamma f(\mathbb{C} \mathbb{E}, \theta))}{1 + k_\phi \int_0^\pi \exp(\gamma f(\mathbb{C}\mathbb{E},\theta')) d \theta'}, \\
& f(\mathbb{C}  \mathbb{E}, \theta)  = \bm{e}_\theta^{\top}   \left (\mathbb{C}  \mathbb{E} \right)  \bm{e}_\theta.
\end{align}
where $c_0, \gamma, k_\phi, k_\rho, k_\rho^{'}$ are suitable positive constants.
\item Fourier formalism:
\begin{align}
& \frac{ d \widehat{\rho_{n}}(t) }{ d t } = \frac{ k_{\rho}  }{ \pi } \frac{ \widehat{\phi_{n} } }{ {\phi_{0} } } - k_{\rho} ^{'} \widehat{\rho_{n} }, \\
& \widehat{\phi_{n}} = \frac{ c_0 }{ \pi } \frac{ I_{n} \left(2 \gamma \left|\widehat{f_1}\right|\right) }{ \frac{ \exp(-\gamma f_0) }{ \pi k_{\phi}  } + I_0 \left( 2 \gamma \left| \widehat{f_1} \right|\right)  } \exp(-2in\theta^*), \\
& I_{n} (x) = \frac{ 1 }{ \pi } \int_{0} ^\pi \exp(x \cos(\theta)) \cos ( n \theta )d \theta \quad \forall x \in \mathbf{R},
\end{align}
where $\theta^*$ is the direction of the main stress.
\end{itemize}
\end{frame}

\begin{frame}{Ongoing}
\begin{itemize}
\item LIU Yuyang: modeling and simulation (consider stress feedback loop model),
\item CUI Bofei: NGSolve \& Netgen toturial.
\item ZHANG Donghang: Meshing method (SCVT \& CVT) for cell division.
\end{itemize}
\end{frame}
\end{document}
